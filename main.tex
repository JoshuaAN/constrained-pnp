\documentclass[12pt]{article}

% Packages
\usepackage{amsmath, amssymb, amsthm} % For math symbols and environments
\usepackage{geometry} % To adjust margins
\geometry{letterpaper, margin=1.5in} % Set page size and margins
\usepackage{enumitem} % For custom lists
\usepackage{fancyhdr} % For header and footer
\usepackage{lastpage} % For the last page number
\usepackage{graphicx}
\usepackage{tcolorbox}
\usepackage{algorithm}
\usepackage{algpseudocode}
\usepackage{mathtools}
\usepackage{mathpazo}
\usepackage{listings}
\usepackage{hyperref}

\usepackage{pgfplots}
\pgfplotsset{compat=1.18}

% Header and Footer
\pagestyle{fancy}
\fancyhf{}
\fancyhead[L]{\textbf{Solving the Constrained Perspective-n-Point Problem}}
\fancyhead[R]{Joshua Nichols}
\fancyfoot[C]{Page \thepage\ of \pageref{LastPage}}

% Custom Commands
\newcommand{\problem}[1]{\subsection*{Problem #1}}
\newcommand{\solution}{\subsubsection*{Solution}}
\newcommand{\PP}{\mathbb{P}}

\newcommand{\parens}[1]{\mathopen{}\left( {#1}_{{}_{}}\,\negthickspace\right)\mathclose{}}
\newcommand{\bracks}[1]{\mathopen{}\left[ {#1}_{{}_{}}\,\negthickspace\right]\mathclose{}}
\newcommand{\braces}[1]{\mathopen{}\left\{ {#1}_{{}_{}}\,\negthickspace\right\}\mathclose{}}
\newcommand{\angles}[1]{\mathopen{}\left\langle {#1}_{{}_{}}\,\negthickspace\right\rangle\mathclose{}}

\setlength{\parindent}{0pt}
\setlength{\parskip}{10pt}

\tcbset{
    colframe=black, % Color of the box border
    colback=white,  % Background color
    sharp corners,  % Box style
    boxrule=0.5pt,  % Thickness of the box border
}

% Begin Document
\begin{document}

\title{Solving the Constrained Perspective-n-Point Problem}
\author{Joshua Nichols}
\date{March 18th 2025}
\maketitle

\section{Introduction}

In this writeup we discuss a version of the perspective-n-point problem with the solution constrained to lie in the SE(2) group. This approach is useful in robotics where we wish to solve for the pose of a robot.

talk about opencv here

Note the notation and problem structure is heavily based on \cite{zhou2019iros}.

\newpage

\section{Problem Formulation}

Suppose we have $N$ image points $\mathbf{p}_i=[x_i,y_i]^\intercal$ and corresponding world points $\mathbf{P}_i=[X_i,Y_i,Z_i]^\intercal$. 
Let $K\in \mathbb{R}^{3\times 3}$ be the camera's intrinsic matrix. 
The objective is to find the 3D pose, represented by the rotation matrix $R\in \mathbb{R}^{3\times 3}$ and translation vector $T\in\mathbb{R}^3$, such that the reprojection error is minimized. 

The additional constraint we introduce is to restrict the pose to lie in the Euclidian group $SE(2)$. With this constraint, the rotation matrix and translation vector can be written as

\begin{equation}
  T=\begin{bmatrix}
    x \\ y \\ 0
  \end{bmatrix}\text{ and }R=\begin{bmatrix}
    \cos \theta & -\sin\theta & 0\\
    \sin \theta & \cos \theta & 0\\
    0 & 0 & 1
  \end{bmatrix},\text{ where }x,y,\theta\in\mathbb{R}.
\end{equation}

We use the Cayley transform to introduce a change of variables

\begin{equation}
R=\begin{bmatrix}
  \cos \theta & -\sin\theta & 0\\
  \sin \theta & \cos \theta & 0\\
  0 & 0 & 1
\end{bmatrix}=\begin{bmatrix}
  \frac{1-\tau^2}{1+\tau^2} & \frac{2\tau}{1+\tau^2} & 0\\
  -\frac{2\tau}{1+\tau^2} & \frac{1-\tau^2}{1+\tau^2} & 0\\
  0 & 0 & 1
\end{bmatrix},\text{ where }\tau\in\mathbb{R}.
\end{equation}

We assume the pinhole camera model, which models the relationship between image points and world points as

\begin{equation}
  s\begin{bmatrix}
    \mathbf{p_i}\\ 1
  \end{bmatrix}=K\begin{bmatrix}
    R, T
  \end{bmatrix}\begin{bmatrix}
    \mathbf{P}_i\\ 1
  \end{bmatrix}, \text{ for some }s\in\mathbb{R}.
\end{equation}

To simplify the problem, we multiply both sides by $K^{-1}$

\begin{equation}
  sK^{-1}\begin{bmatrix}
    \mathbf{p_i}\\ 1
  \end{bmatrix}=\begin{bmatrix}
    R, T
  \end{bmatrix}\begin{bmatrix}
    \mathbf{P}_i\\ 1
  \end{bmatrix}, \text{ for some }s\in\mathbb{R}.
\end{equation}

For the rest of the paper, we will use normalized image points $\mathbf{u_i}=[u_i,v_i]^\intercal$ defined as $\begin{bmatrix}
  \mathbf{u}_i \\ 1
\end{bmatrix}=K^{-1}\begin{bmatrix}
  \mathbf{p_i} \\ 1
\end{bmatrix}$. Rewriting (4) with normalized image points gives

\begin{equation}
  s\begin{bmatrix}
    \mathbf{u_i}\\ 1
  \end{bmatrix}=\begin{bmatrix}
    R, T
  \end{bmatrix}\begin{bmatrix}
    \mathbf{P}_i\\ 1
  \end{bmatrix}, \text{ for some }s\in\mathbb{R}.
\end{equation}

For notational purposes, let $\mathbf{r_i}$ represent the $i$-th row of the rotation matrix and $t_i$ represent the $i$-th element of the translation vector.
From (5) we can solve for $s$

\begin{equation}
  \begin{bmatrix}
    su_i\\sv_i\\ s
  \end{bmatrix}=\begin{bmatrix}
    R, T
  \end{bmatrix}\begin{bmatrix}
    \mathbf{P_i}\\ 1
  \end{bmatrix}=\begin{bmatrix}
    \mathbf{r}_1\mathbf{P}_i+t_1\\\mathbf{r}_2\mathbf{P}_i+t_2\\\mathbf{r}_3\mathbf{P}_i+t_3
  \end{bmatrix}\implies s=\mathbf{r}_3\mathbf{P}_i+t_3.
\end{equation}

Substituting $s=\mathbf{r}_3\mathbf{P}_i+t_3$ into (5) based on the result from (6)  gives

\begin{align}
  u_i\parens{\mathbf{r_3}\mathbf{P}_i+t_3}-\mathbf{r_1}\mathbf{P}_i+t_1=0,\\
  v_i\parens{\mathbf{r_3}\mathbf{P}_i+t_3}-\mathbf{r_2}\mathbf{P}_i+t_2=0.
\end{align}

Substituting the definition of $R$ into (7) and (8) simplifies to

\begin{align}
  u_iZ_i(1+\tau^2)-X_i(1-\tau^2)-Y_i(2\tau)-t_1(1+\tau^2)=0,\\
  v_iZ_i(1+\tau^2)+X_i(2\tau)-Y_i(1-\tau^2)-t_2(1+\tau^2)=0.
\end{align}

Next, we introduce another change of variables

\begin{equation}
  t_i'=t_i(1+\tau^2).
\end{equation}
  
Substituting (10) into (8) and (9) gives

\begin{align}
  E_i^x=u_iZ_i(1+\tau^2)-X_i(1-\tau^2)-Y_i(2\tau)-t'_1=0,\\
  E_i^y=v_iZ_i(1+\tau^2)+X_i(2\tau)-Y_i(1-\tau^2)-t'_2=0.
\end{align}

This reduces the problem such that (12) and (13) are now both second degree polynomials. Factoring out $\tau$ gives

\begin{align}
  E_i^x=(u_iZ_i+X_i)\tau^2-2Y_i\tau+u_iZ_i-X_i-t'_1=0,\\
  E_i^y=(v_iZ_i+Y_i)\tau^2+2X_i\tau+v_iZ_i-Y_i-t'_2=0.
\end{align}

We can now formulate this as a least-squares problem

\begin{equation}
  \min_{t_1',t_2',\tau}\sum_{i=1}^N(E_i^x)^2+(E_i^y)^2.
\end{equation}

To further simplify the problem, we can find a closed form solution for $t_1'$ and $t_2'$ in terms of $\tau$. If we fix $\tau$ as a constant, we have the following optimization problem

\begin{equation}
  \min_{t_1',t_2'}\sum_{i=1}^N(E_i^x)^2+(E_i^y)^2=\min_{t_1'}\sum_{i=1}^N(E_i^x)^2+\min_{t_2'}\sum_{i=1}^N(E_i^y)^2.
\end{equation}

First we solve for $t_1'$. The first order optimality condition is

\begin{equation}
    \frac{d}{dt_1'}\sum_{i=1}^N(E_i^x)^2=0\implies \sum_{i=1}^N\parens{\frac{d}{dt_1'}(E_i^x)^2}=0
\end{equation}

Differentiating by the chain rule gives

\begin{equation}
  \frac{d}{dt_1'}(E_i^x)^2=2\parens{\frac{d}{dt_1'}E_{i}}E_i^x=-2E_i^x.
\end{equation}

Substituting (19) into (18)

\begin{equation}
  \sum_{i=1}^N\parens{-2E_i^x}= \sum_{i=1}^NE_i^x=0.
\end{equation}

Finally solving for $t_1'$ gives

\begin{equation}
  t_1'=\frac{1}{N}\sum_{i=1}^N\parens{(u_iZ_i+X_i)\tau^2-2Y_i\tau+u_iZ_i-X_i}.
\end{equation}

Repeating the same for $t_2'$ gives

\begin{equation}
  t_2'=\frac{1}{N}\sum_{i=1}^N\parens{(v_iZ_i+Y_i)\tau^2+2X_i\tau+v_iZ_i-Y_i}.
\end{equation}

Observe $t_1'$ and $t_2'$ are second degree polynomials of $\tau$. Let us define the constants

\begin{align}
  A_1=\frac{1}{N}\sum_{i=1}^N (u_iZ_i+X_i)\\
  B_1=\frac{1}{N}\sum_{i=1}^N (-2Y_i)\\
  C_1=\frac{1}{N}\sum_{i=1}^N (u_iZ_i-X_i)\\
  A_2=\frac{1}{N}\sum_{i=1}^N (v_iZ_i+Y_i)\\
  B_2=\frac{1}{N}\sum_{i=1}^N (2X_i)\\
  C_2=\frac{1}{N}\sum_{i=1}^N (v_i Z_i-Y_i)
\end{align}

Now we can rewrite the errors as

\begin{align}
  E_i^x=(u_iZ_i+X_i-A_1)\tau^2+(-2Y_i-B_1)\tau+u_iZ_i-X_i-C_1=0,\\
  E_i^y=(v_iZ_i+Y_i-A_2)\tau^2+(2X_i-B_2)\tau+v_iZ_i-Y_i-C_2=0.
\end{align}

Substituting back into our least-squares problem, we now have a single variable optimization problem 

\begin{equation}
  \min_{\tau}C(\tau),\qquad C(t)=\sum_{i=1}^N(E_i^x)^2+(E_i^y)^2.
\end{equation}

Observe $C(\tau)$ is a fourth degree polynomial. Thus the first-order optimality condition is simply a third-degree polynomial of $\tau$. Cardano's formula provides a closed form solution to this polynomial.

The final step is to recover $t_1$, $t_2$, and $\theta$ as

\begin{align}
  t_1=\frac{t_1'}{(1+\tau^2)},\\
  t_2=\frac{t_2'}{(1+\tau^2)},\\
  \theta=2\tan^{-1}(\tau).
\end{align}

Thus we have shown a closed form solution for the constrained perspective-n-point problem.

\newpage 

\bibliographystyle{IEEEtran}
\bibliography{references}

\end{document}
